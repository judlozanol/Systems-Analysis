\section{Key elements of the competition}

\subsection*{Environment observations:}
\begin{itemize}
    \item Ball information: position, direction, rotation, and possession.
    \item Team information: position, direction, fatigue level, cards, and roles.
    \item Controlled player: specifies which player is being controlled at the moment.
    \item Game mode: includes contextual information such as corner kick, penalty, match start, etc.
\end{itemize}

\subsection*{Available actions:}
\begin{itemize}
    \item Up to 19 different actions can be executed, including movement, passes, shots, sprint, dribble, among others. These are detailed in the observation file.
\end{itemize}

\subsection*{Agent configurations:}
\begin{itemize}
    \item Architecture of the model used.
    \item Training algorithm applied.
    \item Observation representation technique: can be raw, simple115\_v2, or pixels.
\end{itemize}

\subsection*{System process:}
\begin{itemize}
    \item To evaluate its performance, each agent automatically pits itself against other agents with comparable skill levels (determined by $\mu$).
    \item Eight matches are played every day. Each agent's score is modified in accordance with:
    \begin{itemize}
        \item The match's outcome (win, draw, or loss).
        \item The variation (based on $\mu$) between the expected and actual result.
        \item The degree of uncertainty ($\sigma$) connected to every agent.
    \end{itemize}
\end{itemize}

\subsection*{System outputs:}
\begin{itemize}
    \item Approximate rating: Denoted by the $\mu$ value of the agent.
    \item Last leaderboard: A list displaying each team's best agent.
    \item Performance history: Record of the performance of all agents.
\end{itemize}