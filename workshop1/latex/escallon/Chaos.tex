\section{Chaos Theory in Football}

A sport like football is largely subject to chance and, therefore, to chaos. In it, a single individual action can completely change the course of the match—such as a foul that leads to a player's expulsion, a counterattack that occurs independently of the game's context, a mistake by a player or even the referee. Even factors like the field and the weather are fundamental elements that do not depend on the players themselves and can seriously affect the outcome of the match.

All these types of situations are quite difficult to transfer into simulation environments, as these are usually carried out under perfect conditions. If such cases were incorporated into simulations, the results could be more efficient and realistic.

In these environments, artificial intelligence can misinterpret data due to random factors. An AI agent might develop a very effective strategy that allows it to control the match for most of the time, but due to one of these isolated events, it could interpret its strategy as ineffective.

Factors such as rebounds strongly influence the outcome of matches, and they are unpredictable, generating dangerous plays that could impact the final result.



